\documentclass[11pt, oneside, a4paper]{article}
\usepackage{ifpdf}
\usepackage[colorlinks,bookmarksopen]{hyperref}
\usepackage{color}
\usepackage{latexsym}
\usepackage{amssymb}
\usepackage{amsmath}
\usepackage{graphicx}
\usepackage{wasysym}
\usepackage{cancel}
\usepackage{german}
\usepackage[utf8]{inputenc}
\begin{document}
\begin{titlepage}
\title{Analysis für Informatiker 2012/13}
\author{bei Prof. Dr. Lau}
\date{October 08, 2012}
\maketitle
\end{titlepage}
\newpage
\tableofcontents
\newpage
\section*{1 Mengen}
\subsection*{1.1 Definition}
\subsubsection*{1)}
Eine Menge ist eine Ansammlung verschiedener Objekte.
\subsubsection*{2)}
Die Objekte einer Menge heißen \underline{Elemente}. \\ \\
\textit{Notation:} \\
$a \in M$ heißt: $a$ ist Element der Menge $M$ \\
$a \notin M$ heißt: $a$ ist kein Element der Menge $M$
\subsubsection*{3)}
Sei $M$ eine Menge. Eine Menge $U$ heißt Teilmenge von $M$, wenn jedes Element von $U$ auch ein Element von $M$ ist. \\
\textit{Notation:} \\
$U \subseteq M$ heißt: $U$ ist Teilmenge der Menge $M$ \\
$U \not \subseteq M$ heißt: $U$ ist keine Teilmenge der Menge $M$
\subsection*{1.2 Beispiele}
\subsubsection*{1)}
Sei $M$ die Menge aller Studierenden in L1, \\
$W$ die Menge aller weiblichen Studierenden in L1, \\
$F$ die Menge aller Frauen. \\ \\
Dann gilt: $W \subseteq M, W \subseteq F, M \not \subseteq F, F \not \subseteq M$.
\subsubsection*{2)}
Die Menge der natürlichen Zahlen: $\mathbb+{N}=\{1,2,3,..\}$ \\ \\
G sei Menge der geraden natürlichen Zahlen. \\
$G := \{n \in \mathbb{N} |$ $n$ gerade $\}$ \\
\hphantom{spaces}$ := \{2m | m \in \mathbb{N}\}$ \\
\hphantom{spaces}$ := \{2,4,6,8,...\}$ \\ \\
Es gilt $G \subseteq \mathbb{N}$, $\mathbb{N} \not \subseteq G$.
\subsubsection*{3)}
Die Menge der ganzen Zahlen: \\
$\mathbb{Z} = \{0,-1,1,-2,2,...\}$
\subsubsection*{4)}
Menge der rationalen Zahlen: \\
$\mathbb{Q} = \frac{a}{b} | a,b \in \mathbb{Z}, b \not = 0\}$
\subsubsection*{5)}
Die Menge ohne Elemente heißt \underline{leere Menge}. \\
\textit{Symbol:} $\emptyset = \{\}$ \\ \\
\framebox{Bemerkung} \\
\hphantom{spaces} Für jede Menge $M$ gilt: $\emptyset \subseteq M$ \\
\hphantom{spaces} $\mathbb{N} \subseteq \mathbb{Z} \subseteq \mathbb{Q}$
\subsection*{1.3 Definition}
Sei $M$ eine Menge, $U,V \subseteq M$ Teilmengen.
\subsubsection*{1)}
Die Vereinigung von $U$ und $V$ ist $U \cup V := \{x \in M | x \in U$ oder $x \in V\}$
\subsubsection*{2)}
Der Durchschnitt von $U$ und $V$ ist $U \cap V := \{x \in M | x \in U$ oder $x \in V\}$
\subsubsection*{3)}
Die Differenzmenge von $U$ und $V$ ist $U \slash V := \{x \in U | x \not \in V\}$
\subsubsection*{4)}
Das Komplement von $U$ ist $U^c = M \slash U = \{x \in M | x \not \in U\}$ \\ \\
\textit{Beispiel} \\
Sei $M = \mathbb{N}$.\\
$\{1,3\} \cup \{3,5\} = \{1,3,5\}$ \\
$\{1,3\} \cap \{3,5\} = \{3\}$ \\
$\{1,3\} \cap \{2,4,7\} = \emptyset$ ,also sind $\{1,3\}$ und $\{2,4,7\}$ disjunkt. \\
$\{1,2,3\} \slash \{3,4,5\} = \{1,2\}$ \\
$\{1,3,5\}^c = \{2,4,6,7,8,...\}$
\subsection*{1.4 (de Morgansche Regeln)}
$M$ Menge, $U,V \subseteq M$ Teilmengen.
\subsubsection*{1)}
$(U \cup V)^c = U^c \cap V^c$
\subsubsection*{2)}
$(U \cap V)^c = U^c \cup V^c$ \\ \\
\subsection*{Beweis}
\subsubsection*{1)}
Sei $x \in M$. Es gilt: \\
$x \in (U \cup V)^c \Leftrightarrow x \not \in U \cup V \Leftrightarrow x \not \in U$ und $x \not \in V$ \\ 
$\Leftrightarrow x \in U^c $ und $x \in V^c \Leftrightarrow x \in U^c \cap V^c$ \\
\hphantom{spaces}\hphantom{spaces}\hphantom{spaces}\hphantom{spaces}\hphantom{spaces}\hphantom{spaces}\hphantom{spaces}$\square$
\subsubsection*{2)}
Übungsaufgabe
\newpage
\framebox{Vollständige Induktion}
\subsection*{1.5 Prinzip der vollständigen Induktion}
Für jedes $n \in \mathbb{N}$ sei eine Aussage $A(n)$ gegeben. \\
Ziel: Beweisen, dass $A(n)$ für jedes $n \in \mathbb{N}$ wahr ist.\\
Dafür reicht es zu zeigen:
\subsubsection*{1)}
Induktionsanfang: $A(1)$ ist wahr. 
\subsubsection*{2)}
Induktionsschritt: Wenn für ein $n \in \mathbb{N}$ $A(n)$ wahr ist, dann ist auch $A(n+1)$ wahr.
\end{document}
